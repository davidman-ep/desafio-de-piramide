\documentclass{article}
\usepackage[utf8]{inputenc}
\usepackage[spanish]{babel}
\usepackage{listings}
\usepackage{graphicx}
\graphicspath{ {images/} }
\usepackage{cite}

\begin{document}

\begin{titlepage}
    \begin{center}
        \vspace*{1cm}
            
        \Huge
        \textbf{Seguimiento de instrucciones}
            
        \vspace{0.5cm}
        \LARGE
        desafió de la pirámide 
            
        \vspace{1.5cm}
            
        \textbf{Rolman David Echavarria Prince}
            
        \vfill
            
        \vspace{0.8cm}
            
        \Large
        Despartamento de Ingeniería Electrónica y Telecomunicaciones\\
        Universidad de Antioquia\\
        Medellín\\
        Marzo de 2021
            
    \end{center}
\end{titlepage}

\tableofcontents
\newpage
\section{Introducción}\label{intro}
Demostrar la capacidad motriz de un ejercicio con una sola mano; buscando formar con dos cartas una pirámide sobre una hoja, el objetivo de ver como el participante logra comprender las instrucciones.

\section{Sección de contenido} \label{contenido}
Desafió de formar una pirámide con dos cartas sobre una hoja.
\subsection{materiales}
\begin{enumerate}
    \item Dos cartas (Preferiblemente iguales).
    \item Una hoja en condiciones normales.
    \item Disponer de una superficie horizontal plana (Preferiblemente una mesa).
\end{enumerate}
\subsection{Sugerencias}
\begin{enumerate}
    \item [-]El ángulo superior que une las dos cartas formen aproximadamente 40º.
    \item[-]Los lados cortos superiores que unen las dos cartas alineadas, quede uno sobre el otro justamente, considerar esto si una carta es más densa que la otra.
\end{enumerate}
\subsection{Citación}
Vamos a citar por ejemplo un artículo de \textbf{Albert Einstein} \cite{einstein}.
También es posible citar libros \cite{dirac} o documentos en línea \cite{knuthwebsite}.\\\\
Revisar en la última sección el formato de las referencias en IEEE.



\section{Inclusión de imágenes} \label{imagenes}

En la Figura (\ref{fig:cpplogo}), se presenta el logo de C++ contenido en la carpeta images.

\begin{figure}[h]
\includegraphics[width=4cm]{cpplogo.png}
\centering
\caption{Logo de C++}
\label{fig:cpplogo}
\end{figure}

Las secciones (\ref{intro}), (\ref{contenido}) y (\ref{imagenes}) dependen del estilo del documento.

\bibliographystyle{IEEEtran}
\bibliography{references}

\end{document}
